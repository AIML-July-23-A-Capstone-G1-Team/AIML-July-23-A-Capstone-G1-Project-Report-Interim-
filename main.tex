\documentclass[11pt]{report} 

\usepackage{projectreport}

% ------------------------------------------------------------------------------
%  Title page 
% ------------------------------------------------------------------------------
\newcommand{\name}{Abhijith A., S. Praneel, Rashmi P., Arijit C., Goverdhan R. G., and Shridurga M.}
\newcommand{\course}{Interim Report}
\newcommand{\projecttitle}{Pneumonia Detection Challenge}
\newcommand{\submissiondate}{July 7 2024}


% ------------------------------------------------------------------------------
% Document
% ------------------------------------------------------------------------------
\begin{document}

\maketitle

% ------------------------------------------------------------------------------
% Top matter
% ------------------------------------------------------------------------------
\chapter*{Abstract}

This report explores pneumonia detection using computer vision techniques on chest radiographs. The study begins with an extensive Exploratory Data Analysis (EDA) to understand the dataset's structure and underlying patterns, including patient demographics and distribution of class labels. EDA findings revealed a balanced distribution of images across various patient ages, sexes, and view positions, although further data augmentation is suggested to enhance the dataset's diversity.

Subsequently, multiple convolutional neural network (CNN) models were developed and evaluated to classify images based on the presence of lung opacity. Model 1, using the RMSprop optimizer, showed moderate performance but indicated potential overfitting. Model 2, utilizing the Adam optimizer, improved initial training stability but faced similar overfitting challenges. Model 3, a simplified architecture, demonstrated better training and validation accuracy after implementing improvement strategies like learning rate reduction and model checkpointing. Despite these efforts, the results highlighted the necessity for further tuning and advanced techniques to achieve optimal performance.

The project transitions into predicting the target variable, with plans to employ Region-based Convolutional Neural Networks (RCNN) in subsequent phases to enhance class label classification. Final model evaluation through confusion matrix and ROC curve analyses underscored the need for balancing sensitivity and specificity, with current metrics indicating areas for improvement.

Future steps involve comprehensive data augmentation, exploration of advanced architectures, and the incorporation of transfer learning to improve model performance.
\vspace{2cm}

\chapter*{Acknowledgements}
We extend our heartfelt gratitude to our mentor, Jayant, for his invaluable guidance throughout this project. His insights and expertise have been instrumental in shaping our work. We also sincerely thank our program manager, Avaneesh, for his constant support and encouragement, which have motivated us to strive for excellence.

We are grateful to Great Learning for facilitating this learning platform and providing the necessary resources and tools. Their commitment to education and continuous support have significantly contributed to the success of our project.

\tableofcontents
\listoffigures
\listoftables

% ------------------------------------------------------------------------------
% Main matter
% ------------------------------------------------------------------------------
\newpage
\setcounter{page}{0}
\pagenumbering{arabic}
% ------------------------------------------------------------------------------
% Chapter 1
% ------------------------------------------------------------------------------
\chapter{Summary of Problem Statement, Data, and Findings}
\label{cha:chapter 1}


\section{Problem Statement}
\label{sec:chap1 section 1}

\section{Data Description}
\label{sec:chap1 section 2}

\section{Initial Findings}
\label{sec:chap1 section 3}

\section{Conclusion}
\label{sec:chap1 section 4}

% ------------------------------------------------------------------------------
% Chapter 2
% ------------------------------------------------------------------------------
\chapter{Summary of the Approach to EDA and Pre-processing}
\label{cha:chapter 2}

\section{Exploratory Data Analysis (EDA)}
\label{sec:chap2 section 1}

\section{Pre-processing}
\label{sec:chap2 section 2}


% ------------------------------------------------------------------------------
% Chapter 3
% ------------------------------------------------------------------------------
\chapter{Deciding Models and Model Building}
\label{cha:chapter 3}

\section{Model Selection}
\label{sec:chap3 section 1}

\section{Model Building}
\label{sec:chap3 section 2}

% ------------------------------------------------------------------------------
% Chapter 4
% ------------------------------------------------------------------------------
\chapter{How to Improve Your Model Performance?}

\section{Improvement Strategies}
\label{sec:chap4 section 1}

After initial training, Model 3 showed promising results but still left room for improvement. To enhance the model's performance, several fine-tuning strategies were implemented. The ReduceLROnPlateau callback was used to monitor the validation accuracy, and the learning rate was reduced if the accuracy did not improve over ten epochs. This approach helped the model to converge more effectively by adjusting the learning rate dynamically. Additionally, ModelCheckpoint was utilized to save the best version of the model based on the minimum validation loss, ensuring that the most optimal weights were retained. Figure ~\ref{fig:cha-4 figure1} shows the training and validation accuracy and loss plots after implementing these strategies.

\begin{figure}
    \centering
    \includegraphics[width=0.8\textwidth]{figures/Figure32.png}
    \caption{Training and Validation Accuracy and Loss Plots}
    \label{fig:cha-4 figure1}
\end{figure}

These fine-tuning efforts led to noticeable improvements in both training and validation accuracy and loss. As depicted in the accompanying plots, the training accuracy steadily increased, and the validation accuracy showed a more stable trend compared to the initial runs. The training loss decreased significantly, and although the validation loss exhibited some fluctuations, the overall trend was positive, indicating a better generalization capability of the model.

\section {Model Evaluation}
\label{sec:chap4 section 2}
With the best model, we computed the confusion matrix and ROC curve. These are given in Figure ~\ref{fig:cha-4 figure2} and Figure ~\ref{fig:cha-4 figure3} respectively. The confusion matrix and ROC curve provide insights into the model's performance. The confusion matrix indicates that while the model correctly identifies a significant number of true positives (6154) and true negatives (12550), there is still a substantial number of false positives (3951) and false negatives (1492). This suggests that the model might benefit from further fine-tuning and potential adjustments in decision thresholds. The ROC curve, with an area under the curve (AUC) of 0.78, reflects a moderate level of discrimination between the positive and negative classes. While this is a promising start, aiming for a higher AUC by refining the model and its parameters will be crucial in improving overall accuracy and reliability.

\begin{figure}
    \centering
    \includegraphics[width=0.8\textwidth]{figures/Figure33.png}
    \caption{Confusion Matrix for the Best Model}
    \label{fig:cha-4 figure2}
\end{figure}

\begin{figure}
    \centering
    \includegraphics[width=0.8\textwidth]{figures/Figure34.png}
    \caption{ROC Curve for the Best Model}
    \label{fig:cha-4 figure3}
\end{figure}

\section{Future Steps}
\label{sec:chap4 section 3}

To further enhance the model's performance and prevent overfitting, several additional strategies can be considered (These are not implemented in the current model but can be explored in future iterations if time and resources permit):

\begin{itemize}
    \item \textbf{Data Augmentation:} Implementing advanced data augmentation techniques such as random rotations, translations, and horizontal flipping can help increase the diversity of the training data, making the model more robust.

    \item \textbf{Regularization:} Adding L2 regularization to the convolutional and dense layers can help prevent overfitting by penalizing large weights and encouraging simpler models.

    \item \textbf{Dropout Adjustment:} Experimenting with different dropout rates in the network can help determine the optimal level of regularization, ensuring that the model does not rely too heavily on any particular set of neurons.

    \item \textbf{Batch Normalization:} Integrating batch normalization layers can stabilize and accelerate the training process by normalizing the inputs to each layer, improving both training speed and model performance.

    \item \textbf{Learning Rate Scheduling:} Implementing advanced learning rate scheduling techniques such as cosine annealing or cyclic learning rates can help the model converge more efficiently and avoid local minima.

    \item \textbf{Ensemble Methods:} Training multiple models with different architectures and combining their predictions can lead to more robust and accurate results by leveraging the strengths of each model.

    \item \textbf{Transfer Learning:} Utilizing pre-trained models on similar datasets can provide a strong starting point, allowing the model to leverage learned features and improve performance with less training data.

    \item \textbf{Hyperparameter Tuning:} Conducting extensive hyperparameter tuning using techniques such as grid search or Bayesian optimization can help identify the optimal configuration for the model, leading to better performance.
\end{itemize}

% ------------------------------------------------------------------------------
% Reference list
% ------------------------------------------------------------------------------
\addcontentsline{toc}{chapter}{References}
\renewcommand\bibname{References}
\printbibliography

% ------------------------------------------------------------------------------
% Appendices
% ------------------------------------------------------------------------------
% --------------------------------------------------
% Appendices
% --------------------------------------------------
\appendix
\chapter{Code Snippets}

\section{Appendix section}


\chapter{Additional Visualizations}

\end{document}