% --------------------------------------------------
% Appendices
% --------------------------------------------------
\appendix
\chapter{Code Snippets}
\label{app:appendix A}

\section{Image Preprocessing}
\label{app:app-A section1}

\begin{minted}{python}
# Define a function to read the DICOM images and resize them
def readAndReshapeDicomImages(data, image_size):
    patientIds = []
    labels = []
    images = []
    targets = []
    for i, row in notebook.tqdm(data.iterrows(), total=data.shape[0]):
        patientId = row["patientId"]
        label = row["class"]
        target = row["Target"]
        imagePath = row["path"]
        data_row_img = dcm.dcmread(imagePath)
        img = data_row_img.pixel_array
        # Convert the images to 3 channels as DICOM image pixel does not have color channels
        if len(img.shape) != 3 or (len(img.shape) == 3 and img.shape[2] != 3):
            img = np.stack((img,) * 3, -1)
        img = img - np.min(img)
        img = img / np.max(img)
        img = (img * 255).astype(np.uint8)
        img = cv2.resize(img, (image_size, image_size), interpolation=cv2.INTER_AREA)
        patientIds.append(patientId)
        labels.append(label)
        images.append(img)
        targets.append(target)
    return np.array(patientIds), np.array(labels), np.array(targets), np.array(images)
\end{minted}

\section{Model 1 Architecture}
\label{app:app-A section2}

\begin{minted}{python}
model1 = Sequential()

# Define the input shape
model1.add(Input(shape=(HEIGHT, WIDTH, NUM_CHANNELS)))

# First convolutional block
model1.add(Conv2D(filters=32, kernel_size=(5, 5), padding="Same", activation="relu"))
model1.add(Conv2D(filters=32, kernel_size=(5, 5), padding="Same", activation="relu"))
model1.add(MaxPooling2D(pool_size=(2, 2)))
model1.add(Dropout(0.2))

# Second convolutional block
model1.add(Conv2D(filters=64, kernel_size=(3, 3), padding="Same", activation="relu"))
model1.add(Conv2D(filters=64, kernel_size=(3, 3), padding="Same", activation="relu"))
model1.add(MaxPooling2D(pool_size=(2, 2), strides=(2, 2)))
model1.add(Dropout(0.3))

# Third convolutional block
model1.add(Conv2D(filters=128, kernel_size=(3, 3), padding="Same", activation="relu"))
model1.add(Conv2D(filters=128, kernel_size=(3, 3), padding="Same", activation="relu"))
model1.add(MaxPooling2D(pool_size=(2, 2), strides=(2, 2)))
model1.add(Dropout(0.4))

# Global Max Pooling
model1.add(GlobalMaxPooling2D())

# Fully connected layer
model1.add(Dense(256, activation="relu"))
model1.add(Dropout(0.5))

# Output layer
model1.add(Dense(NUM_CLASSES, activation="softmax"))

optimizer = RMSprop(learning_rate=0.001, rho=0.9, epsilon=1e-08)

# Compile model
model1.compile(
    optimizer=optimizer, loss="categorical_crossentropy", metrics=["accuracy"]
)

\end{minted}

\section{Model 2 Architecture}
\label{app:app-A section3}

\begin{minted}{python}
model2 = Sequential()

# Define the input shape
model2.add(Input(shape=(HEIGHT, WIDTH, NUM_CHANNELS)))

# First convolutional block
model2.add(Conv2D(filters=32, kernel_size=(5, 5), padding="Same", activation="relu"))
model2.add(Conv2D(filters=32, kernel_size=(5, 5), padding="Same", activation="relu"))
model2.add(MaxPooling2D(pool_size=(2, 2)))
model2.add(Dropout(0.2))

# Second convolutional block
model2.add(Conv2D(filters=64, kernel_size=(3, 3), padding="Same", activation="relu"))
model2.add(Conv2D(filters=64, kernel_size=(3, 3), padding="Same", activation="relu"))
model2.add(MaxPooling2D(pool_size=(2, 2), strides=(2, 2)))
model2.add(Dropout(0.3))

# Third convolutional block
model2.add(Conv2D(filters=128, kernel_size=(3, 3), padding="Same", activation="relu"))
model2.add(Conv2D(filters=128, kernel_size=(3, 3), padding="Same", activation="relu"))
model2.add(MaxPooling2D(pool_size=(2, 2), strides=(2, 2)))
model2.add(Dropout(0.4))

# Global Max Pooling
model2.add(GlobalMaxPooling2D())

# Fully connected layer
model2.add(Dense(256, activation="relu"))
model2.add(Dropout(0.5))

# Output layer
model2.add(Dense(NUM_CLASSES, activation="softmax"))

# Optimizer
optimizer2 = Adam(learning_rate=0.001, epsilon=1e-08, decay=0.0)

# Compile model
model2.compile(
    optimizer=optimizer2, loss="categorical_crossentropy", metrics=["accuracy"]
)
\end{minted}

\section{Model 3 Architecture}
\label{app:app-A section4}
\begin{minted}{python}
model3 = Sequential()
model3.add(Input(shape=(HEIGHT, WIDTH, NUM_CHANNELS)))
model3.add(Conv2D(filters=32, kernel_size=(3, 3), activation="relu"))
model3.add(MaxPooling2D(pool_size=(2, 2)))
model3.add(Dropout(0.25))
model3.add(Conv2D(filters=64, kernel_size=(3, 3), activation="relu"))
model3.add(MaxPooling2D(pool_size=(2, 2)))
model3.add(Dropout(0.25))
model3.add(GlobalMaxPooling2D())
model3.add(Dense(128, activation="relu"))
model3.add(Dropout(0.5))
model3.add(Dense(NUM_CLASSES, activation="softmax"))
model3.compile(optimizer="adam", loss="categorical_crossentropy", metrics=["accuracy"])
\end{minted}

\section{Fine tuning the model}
\label{app:app-A section5}

\begin{minted}{python}
lrr = ReduceLROnPlateau(
    monitor="val_accuracy", patience=10, verbose=1, factor=0.5, min_lr=0.00001
)
\end{minted}

\section{Appendix section}


\chapter{Additional Visualizations}