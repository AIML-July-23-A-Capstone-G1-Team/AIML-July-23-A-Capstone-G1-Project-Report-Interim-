% ------------------------------------------------------------------------------
% Chapter 2
% ------------------------------------------------------------------------------
\chapter{Summary of the Approach to EDA and Pre-processing}
\label{cha:chapter 2}

In this chapter, we perform an exploratory data analysis (EDA) to gain a deeper understanding of the dataset and uncover insights that will guide the development of our pneumonia detection algorithm. EDA is a critical step in the data science process, as it helps to identify patterns, detect anomalies, and test hypotheses. By visualizing and summarizing the dataset, we can better understand its structure, identify potential issues, and make informed decisions about data preprocessing and model selection.

\section{Exploratory Data Analysis (EDA)}
\label{sec:chap2 section 1}

\subsection{Overview of the Dataset:}
\label{subsec:chap2 section 1.1}

\subsection{Distribution of Classes:}
\label{subsec:chap2 section 1.2}

\subsection{Analysis of Bounding Boxes}
\label{subsec:chap2 section 1.3}

\subsection{Patient Demographics and Data Diversity}
\label{subsec:chap2 section 1.4}

\subsection{Visualization of Images}
\label{subsec:chap2 section 1.5}

\subsection{Handling Missing and Null Values}
\label{subsec:chap2 section 1.6}

\section{Pre-processing}
\label{sec:chap2 section 2}

\subsection{Data Cleaning}
\label{subsec:chap2 section 2.1}

\subsection{Addressing Class Imbalance}
\label{subsec:chap2 section 2.2}

\subsection{Data Augmentation}
\label{subsec:chap2 section 2.3}

\subsection{Splitting the Dataset}
\label{subsec:chap2 section 2.4}

\section{Insights and Observations}
\label{sec:chap2 section 3}

\subsection{Summary of EDA Findings}
\label{subsec:chap2 section 3.1}

\subsection{Impact on Model Development}
\label{subsec:chap2 section 3.2}
